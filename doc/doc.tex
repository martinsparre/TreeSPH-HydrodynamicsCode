%%%%%%%%%S%%%%%%%%%%%%%%%%% Præambel %%%%%%%%%%%%%%%%%%%%%%%%%%%
\typeout{---------- Preamble start -------------} % vises i log

%Dokumenttype og skriftstrrelsSe sættes:
\documentclass[11pt,a4paper]{article} %fjern draft i færdigt dokument. Brug 11pt eller 12pt

% Oplysninger til article-klassen:
\title{TreeSPH}
\author{Martin Sparre}
\date{2008-2009}

%linjeafstand:
%\linespread{1.3}

%Laveste niveau i indholdsfortegnelse:
\setcounter{tocdepth}{2}
%Numereringsniveau:
\setcounter{secnumdepth}{2}

%Ligningsnummerering tager formen (section . nr)
%\numberwithin{equation}{section}


%ingen indrykning ved nyt afsnit:
%\parindent=0pt

%Der vises labels i margin - br kun anvendes i draft
%\usepackage[notref,notcite]{showkeys}

%klikbar indholdsfortegnelse m.m.
%\usepackage{hyperref}

%%%%%%%%%%%%%%%%%%%%%%%%%%% Vigtige Pakker %%%%%%%%%%%%%%%%%%%%%%%%%%%%%

%natbib - en bibTeX-pakke. [Curly] gr, at der bruges {} ved citationer. Natbib skal være før babel
%\usepackage[curly]{natbib}

% En række pakker, der skal inkluderes, hvis man skriver på dansk
\usepackage[utf8]{inputenc} %Danske bogstaver kan bruges, ansinew for windoze - latin1 for linux
\usepackage[danish]{babel} %Kapitler mv. får danske navne
\usepackage[T1]{fontenc}

%Font:
%\usepackage{lmodern}
%\usepackage{palatino}


%\usepackage{sectsty}
%\allsectionsfont{\large\sffamily}

%Marginer:
%\usepackage{vmargin}
%\setpapersize{A4}
%argumenter:{hleftmargini}{htopmargini}{hrightmargini}{hbottommargini}....
%\setmarginsrb{30mm}{20mm}{30mm}{20mm}{12pt}{11mm}{0pt}{11mm}

%fancyhdr - sidehoved og sidefod
%\usepackage{fancyhdr}
%Indholdet af sidehoved og sidefod:
%\renewcommand{\headheight}{14.5pt} %er obligatorisk v. fancyhdr
%\renewcommand{\headrulewidth}{0.5pt}
%\renewcommand{\footrulewidth}{0.5pt}
%\lhead{Martin Sparre} \chead{Titel} \rhead{02-07-2006}
%\lfoot{bla}
%\cfoot{Side \thepage~af \pageref{lastpage}}
%\rfoot{bla}

%En række ams-pakker til matematik:
\usepackage{amsfonts,amsmath,amssymb}

%ntheorem
%\usepackage[amsmath,thmmarks]{ntheorem}
%\theoremsymbol{$\vartriangleleft$}
%\theoremheaderfont{\bfseries\sffamily}
%\theorembodyfont{\normalfont}
%\newtheorem{st}{Sætning}
%\newtheorem{definition}{Definition}
%\newtheorem{eksempel}{Eksempel}
%\newtheorem{Metode}{Metode}

% Pakke til at lave SI-enheder ved \unit{tal}{enhed}. - "squaren" er tilfjet for at undgå konflikt med amssymb.
%\usepackage[squaren]{SIunits} 
%Enheder der tilfjes:
%\addunit{\molaer}{M}

%inkludering af kildekode:
%\usepackage{verbatim}

%overfull hbox under 3pt ignoreres:
\hfuzz=4pt


%%%%%%%%%%%%%%%%%%%%%%%%%%%%%%%%%%%%%%%%%%%%%%%%%%%%%
%%%%%%%% Pakker til inkludering af grafik: %%%%%%%%%%
%%%%%%%%%%%%%%%%%%%%%%%%%%%%%%%%%%%%%%%%%%%%%%%%%%%%%

%\usepackage{graphicx} 

%pakke så floats ikke flyder, indsæt [H] som argument efter \begin{figur}.
%\usepackage{float}

%Denne pakke srger for at floats ikke flyder ind i andre afsnit ved kommandoen \FloatBarrier (dette medfrer ikke ny side i modsætning til \clearpage)
%\usepackage{placeins}

%labelfonts i captions bliver fede og captionmargin forstrres:
%\usepackage{caption}
%\captionsetup{font=small,labelfont=bf}
%\setlength{\captionmargin}{20pt}



%%%%%%%%%%%%%%%%%%%%% NYE KOMMANDOER %%%%%%%%%%%%%%%%%%%%%%%%%%%%%%

\newcommand{\mc}[1]{\mathcal{#1}}
\newcommand{\mb}[1]{\mathbf{#1}}

%Horizontale tykke streger via \HRule :
\newcommand{\HRule}{\rule{\textwidth}{1mm}}

%d'er som bruges ved infinitesimaler:
\DeclareMathOperator{\di}{d\!}

%orddeling
%\hyphenation{Mar-tin}


%%%%%%%%%%%%%%%%%%%%%%%%%%%%%%%%%%%%%%%%%%%%%%%%%%%%%%%%%%%%%%%%%%%
%%%%%%%%%%%%%%%%%%%%% DOKUMENT START %%%%%%%%%%%%%%%%%%%%%%%%%%%%%%
%%%%%%%%%%%%%%%%%%%%%%%%%%%%%%%%%%%%%%%%%%%%%%%%%%%%%%%%%%%%%%%%%%%

\typeout{---------- Dokument start -------------} % vises i log
\begin{document}
\typeout{-------------- forside ----------------}
%\maketitle


%\tableofcontents
\typeout{---------- START -------------} % vises i log
\section*{Leapfrog integration}
\subsection*{The variables}
The file with the initial conditions contains the following quantities for each of the $N$ particles:
\begin{align*}
&\text{Position: }\mb{r}_i,\\
&\text{Velocity: }\mb{v}_i,\\
&\text{Energy per mass unit: } u_i,\\
&\text{Mass: } m_i.
\end{align*}
At every timestep the smoothing length, $h_i$, of each particle is calculated such that the number of
particles inside a sphere of radius $h_i$ is constant.
 
When the positions are known the density of each particle can be calculated:
\begin{align*}
\rho_i=\rho_i (\mb{r}).
\end{align*}
$\rho_i (\mb{r}) $ is shorthand notation for $\rho_i (\mb{r}_1,\mb{r}_2,\ldots ,\mb{r}_N ) $. The pressure and the entropy are given by
\begin{align*}
P_i&=(\gamma-1)\rho_i u_i,\\
A_i &= \frac{P_i}{\rho_i^\gamma}.
\end{align*} 
The acceleration and the entropy derivative of each particle can be written as
\begin{align*}
\mb{a}_i = \mb{a}_i (\mb{r},\mb{v},\rho ,A ),
\quad \dot{A}_i = \dot{A}_i(\mb{r},\mb{v},\rho ,A ).
\end{align*}

\subsection*{Leapfrog integration scheme}
We want to determine the time-evolution of the variables, $\mb{r},\mb{v}$ and $A$. At the 0th timestep (i.e. at $t=0$) the following quantities are known/calculated:
\begin{align*}
\mb{r}_i^0 &\text{ is known},\\
\mb{v}_i^0&\text{ is known},\\
A_i^0&\text{ is known}, \\
\rho_i^0&=\rho_i^0 (\mb{r}^0), \\
\mb{a}_i^0 &= \mb{a}_i^0 (\mb{r}^0,\mb{v}^0,\rho^0 ,A^0 ),\\
\dot{A}_i^0 &= \dot{A}_i^0(\mb{r}^0,\mb{v}^0,\rho^0 ,A^0 ).
\end{align*}
The superscripts indicate the number of the timestep. First we calculate the velocity and the
entropy at $t=\Delta t/2$:
\begin{align*}
\mb{v}_i^{1/2} &= \mb{v}_i^0 + \mb{a}_i^0\cdot  \Delta t/2,\\
A_i^{1/2} &= A_i^{0}+ \dot{A}_i^{0}\cdot \Delta t/2 .
\end{align*}
Next we calculate the acceleration and the entropy derivative for each particle at $t=\Delta t$ using the
following algorithm:
\begin{align*}
\mb{r}_i^1 &= \mb{r}_i^0 + \mb{v}_i^{1/2} \Delta t,\\
\rho_i^1 &= \rho_i^1 (\mb{r}^1),\\
\tilde{\mb{ v}}_i^1 &= \mb{v}_i^{1/2} + \mb{a}_i^{0}  \Delta t/2 ,\\
\tilde{A}_i^1 &= A^{1/2}_i + \dot{A}_{i}^{0}  \Delta t/2,\\
\mb{a}_i^1 &= \mb{a}_i^1 ( \mb{r}^1, \tilde{\mb{v}}^1, \rho^1, \tilde{A}^1 ),\\
\dot{A}_i^1 &= \dot{A}_i^1 (\mb{r}^1 ,\tilde{\mb{ v}}^1, \rho^1, \tilde A^1 ).
%\tilde{\tilde{\mb{ v}}}_i^1 &= \mb{v}_i^{1/2} + \frac{1}{2}[\mb{a}_i^{0}+\tilde{\mb{a}}_i^{1} ]  \Delta t/2, \\
%\tilde{\tilde{A}}_i^1 &= A^{1/2}_i + \frac{1}{2}\left[\dot{A}_{i}^{0} + \tilde{\dot{A}}_{i}^{1} \right]  \Delta t/2,\\
%\mb{a}_i^1 &= \mb{a}_i^1 ( \mb{r}^1, \tilde{\tilde{\mb{v}}}^1, \rho^1, \tilde{\tilde{A}}^1 ),\\
%\dot{A}_i^1 &=A_i^1 ( \mb{r}^1, \tilde{\tilde{\mb{v}}}^1, \rho^1, \tilde{\tilde{A}}^1 ).
\end{align*}
Then the velocity and the entropy are computed at $t=3/2 \Delta t$:
\begin{align*}
\mb{v}_i^{3/2} &= \mb{v}_i^{1/2} + \mb{a}_i^1  \Delta t/2,\\
A_i^{3/2} &= A_i^{1/2}+ \dot{A}_i^{1} \Delta t/2 .
\end{align*}
The general algorithm is
\begin{align*}
\mb{r}_i^{n+1} &= \mb{r}_i^n + \mb{v}_i^{n+1/2} \Delta t,\\
\rho_i^{n+1} &= \rho_i^{n+1} (\mb{r}^{n+1}),\\
\tilde{\mb{ v}}_i^{n+1} &= \mb{v}_i^{n+1/2} + \mb{a}_i^{n}  \Delta t/2 ,\\
\tilde{A}_i^{n+1} &= A^{n+1/2}_i + \dot{A}_{i}^{n}  \Delta t/2,\\
\mb{a}_i^{n+1} &= \mb{a}_i^{n+1} ( \mb{r}^{n+1}, \tilde{\mb{v}}^{n+1}, \rho^{n+1}, \tilde{A}^{n+1} ),\\
\dot{A}_i^{n+1} &= \dot{A}_i^{n+1} (\mb{r}^{n+1} ,\tilde{\mb{ v}}^{n+1}, \rho^{n+1}, \tilde A^{n+1} ),\\
%\tilde{\tilde{\mb{ v}}}_i^{n+1} &= \mb{v}_i^{n+1/2} + \frac{1}{2}[\mb{a}_i^{n}+\tilde{\mb{a}}_i^{n+1} ]  \Delta t/2, \\
%\tilde{\tilde{A}}_i^{n+1} &= A^{n+1/2}_i + \frac{1}{2}\left[\dot{A}_{i}^{n} + \tilde{\dot{A}}_{i}^{n+1} \right]  \Delta t/2,\\
%\mb{a}_i^{n+1} &= \mb{a}_i^{n+1} ( \mb{r}^{n+1}, \tilde{\tilde{\mb{v}}}^{n+1}, \rho^{n+1}, \tilde{\tilde{A}}^{n+1} ),\\
%\dot{A}_i^{n+1} &=A_i^{n+1} ( \mb{r}^{n+1}, \tilde{\tilde{\mb{v}}}^{n+1}, \rho^{n+1}, \tilde{\tilde{A}}^{n+1} ),\\
\mb{v}_i^{n+3/2} &= \mb{v}_i^{n+1/2} + \mb{a}_i^{n+1}  \Delta t/2,\\
A_i^{n+3/2} &= A_i^{n+1/2}+ \dot{A}_i^{n+1} \Delta t/2 .
\end{align*}



\label{lastpage}
\end{document}
